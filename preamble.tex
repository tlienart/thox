% !TEX root = ./thesis.tex

% DOC
\documentclass[openright,a4paper,12pt, twoside]{report}
\usepackage[
	includehead,
	hmargin={3.6cm, 2.6cm},
	vmargin={2.5cm,2.7cm},
	headsep=.8cm,
	footskip=1.2cm]{geometry}

% LINE SPACING
\usepackage{setspace}

% HEADERS
\usepackage{fancyhdr}
\setlength{\headheight}{15pt}
\fancyhf{} % clear the header and footers
\pagestyle{fancy}
\renewcommand{\chaptermark}[1]{\markboth{\thechapter. #1}{\thechapter. #1}}
\renewcommand{\sectionmark}[1]{\markright{\thesection. #1}}
\renewcommand{\headrulewidth}{0pt}
\fancyhead[LO]{\emph{\leftmark}}
\fancyhead[RE]{\emph{\rightmark}}
\fancyhead[RO,LE]{\emph{\thepage}}

% PART PAGE
\makeatletter
\renewcommand\part{%
  \if@openright
    \cleardoublepage
  \else
    \clearpage
  \fi
  \thispagestyle{empty}%   % Original »plain« replaced by »emptyx
  \null\vfil
  \secdef\@part\@spart}
\makeatother

% HYPERREF
\usepackage[colorlinks=true,
			linkcolor=TitleCol,
			citecolor=TitleCol,
			urlcolor=TitleCol,
			linktocpage=true,
			pdfstartview=FitH]{hyperref}

% FONT
\usepackage[english]{babel}
\usepackage[utf8x]{inputenc}
\usepackage[T1]{fontenc}
\usepackage[default]{sourcesanspro}

% COLORS
\usepackage[usenames,dvipsnames,svgnames]{xcolor}

\definecolor{TitleCol}{HTML}{28316C} % sections, ...
\definecolor{EnvCol}{HTML}  {28316C} % theorem, ...
\definecolor{CapCol}{HTML}  {28316C} % fig, ...

% CAPTIONS
\usepackage[width=.8\textwidth,					% caption width
			labelfont={bf,sf,color=CapCol},		% caption style
			font={footnotesize},
			format={hang},
			textfont={it}]{caption}

% CHAP STYLE
\usepackage{titlesec}
\makeatletter 			% section & subsection number in margin
	\def\@seccntformat#1{
		\protect\makebox[0pt][r]{
			\csname the#1\endcsname\quad
			}
		}
\makeatother

\titleformat*{\section}{
	\Large\bfseries\selectfont\color{TitleCol}}
\titleformat*{\subsection}{
    \large\bfseries\selectfont\color{TitleCol}}
\titleformat*{\subsubsection}{
    \small\bfseries\selectfont\color{TitleCol}}

% BIBLIOGRAPHY
\usepackage[authoryear,sort]{natbib}

% TIKZ
\usepackage{tikz}
\usetikzlibrary{arrows,shapes}

% MATHS
\usepackage{amsmath}
\usepackage{amssymb}
\usepackage{amsthm}

\numberwithin{equation}{chapter}
