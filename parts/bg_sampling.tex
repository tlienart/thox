% !TEX root = ../thesis.tex

\section{Monte Carlo estimators}

\subsection{\dred{Review heavily}}

In the first case,\add{what first case} one is interested in computing the expected value of a test function $\varphi$ with respect to a distribution $\pi:\mathbb R^{d} \to \mathbb R_+$ i.e.: $I:=\E_\pi[\varphi(X)]$. Assuming it can't be computed analytically, a general approach is to consider a quadrature rule of the form:
%
\eqa{
	I\esp\approx\esp \widehat I_n &=& \sum_{i=1}^{n} w(X^{(i)}) \varphi(X^{(i)})
}
%
for some points $X^{(i)}\in\mathcal X$ and corresponding weights $w(X^{(i)})\in\mathbb R_+$. In low dimensions, we can consider deterministic quadrature rules such as the Gauss-Hermite quadrature. However, as the dimensionality increases, these deterministic rules tend to perform poorly even for a large number of integration points (``curse of dimensionality''). In that case, one can fall back to Monte Carlo integration with
%
\eqa{	
	X^{(i)}\esp \simiid\esp \pi(\cdot),\quad\text{and}\quad w(X^{(i)})\spe n^{-1}. 	
}
%
The strong law of large numbers then indicates that $\widehat I_n\to I$ almost surely with approximation error scaling like $\mathcal O(n^{-1/2})$ independently of the dimensionality of the problem. This is to be compared with deterministic rules which typically have approximation error scaling like $\mathcal O(n^{-\alpha/d})$ for a fixed $\alpha>0$ depending on the quadrature rule.

The problem then becomes one of drawing iid.\ samples from $\pi$. Often, this is also a hard problem and one can consider \emph{importance sampling} (IS)\margnote{IS} whereby samples are drawn from a \emph{proposal distribution}\margnote{proposal} $q\approx \pi$ and the quadrature weights are adjusted to reflect that:
%
\eqa{
	X^{(i)}\esp \simiid\esp q(\cdot), \quad\text{and}\quad w(X^{(i)})\spe {\pi(X^{(i)})\over n q(X^{(i)})}.
} 
%
Provided the support of $q$ englobes that of $\pi$ i.e., $\pi(x)>0\Rightarrow q(x)>0$ and provided that $\E_q[\abs{\varphi(X)}w(X)]$ is bounded, the importance sampling estimator is unbiased and consistent.\check{apr1}


