% !TEX root = ../thesis.tex

\chapter*{Notations}

\todofr{
    \begin{itemize}\itsep0
    	\item see later if should be more formal like Jeremy, for now good to have a list of stuff
    	\item pre-messages are actually confusing (notations $st$)
    	\item a lot of notations are not self sufficient, think of text as in JH ($\rightarrow$ would probably be nice although maybe shorter)
    	\item red highlights unclear or insufficient
		\item sort style between notations and abbreviations (section/chapter)
    \end{itemize}
}

\dred{note} 
\add{explain notations for message, explain add comma for t-1,t}

\setlength{\tabcolsep}{12pt}
\renewcommand{\arraystretch}{1.2}
\begin{tabular}{ll}
\textbf{symbol} 		& \textbf{meaning}\\[.3cm]
$\mathcal X$ 			& sample space\\
$x,y,z \in \mathcal X$	& sampling points\\
$X,Y,Z$ 				& random variables\\
$x_{1:T}$ 				& collection $\{x_{t}\}_{t=1}^{T}$\\
$X \sim q(\cdot)$ 		& random variable drawn from a distribution $q$\\
$\mathcal N(\,\cdot\,; \mu, \Sigma)$ 
						& Normal distribution with mean $\mu$ and covariance matrix $\Sigma$\\
%$p(x_{1:T})$ & joint distribution over the collection $\{x_{t}\}_{t=1}^{T}$\\
$p(x\st y)$ 				& conditional distribution at $x$ given that $y$ was observed\\
$\text{supp}(p)$ 		& set of $x$ such that $p(x)>0$ where $p$ is a distribution\\
$\delta_{a}(x)$ 		& Dirac-delta distribution with point mass at $x=a$\\
$\hat p(x)$ 				& \dred{particle estimator of the distribution $p$}\\
%$\pd_t(x_t)$ & \dred{(in HMM) predictive density $p(x_t\st y_{1:t-1})$}\\
%$\bif_t(x_t)$ & \dred{(in HMM) backward information filter $p(y_{t:T}\st x_t)$}\\
%$\tbif_t (x_t)$ & \dred{(in HMM) normalised backward information filter $\gamma_t(x_t)p(y_{t:T}\st x_t)$}\\
$\psi_{u}$ 				& singleton potential on a node $u$\\
$\psi_{uv}$ 				& pairwise potential on an edge between a node $u$ and a node $v$\\
$\partial i$ 			& neighbourhood of node $i$ in a graph\\
$m_{st}$ 				& message from node $s$ to node $t$\\
$M_{st}$ 				& \dred{pre-message at $s$ going to $t$ in Belief Propagation}\\
$\KL{p,q}$ 				& Kullback-Leibler divergence between two distributions $p$ and $q$\\
$\mathcal F_\phi$ 		& exponential family with sufficient statistic $\phi$\\
$q_\theta$ 				& distribution parametrised by $\theta$\\
$A(\nu)$ 				& log-partition function of an $\mathcal F_{\phi}$ distribution parametrised by $\theta$\\
$A^{\star}(\nu)$		& \dred{convex-conjugate of $A$}\\
%$\theta[q]$ & natural parameter of distribution $q\in \mathcal F_{\phi}$\\
%$\mu[q]$ & mean parameter of distribution $q\in \mathcal F_{\phi}$\\
\dred{$\mathcal P_\phi[p]$} 	& Kullback-Leibler projection of a distribution $p$ on $\mathcal F_\phi$\\
$\E_p[\phi]$ 			& expected value of $\phi(X)$ if $X\sim p(\cdot)$\\
$\V_{p}[\phi]$ 			& \dred{variance of $\phi(X)$} if $X\sim p(\cdot)$  

\end{tabular}
\setlength{\tabcolsep}{6pt} % default

\section*{Abbreviations}
\add{add reference of primary definition, may be nice}
\setlength{\tabcolsep}{12pt}
\renewcommand{\arraystretch}{1.2}
\begin{tabular}{ll}
\textbf{abbreviation} & \textbf{meaning}\\[.3cm]
BIF 	& Backward Information Filter\\
BP/LBP 	& (Loopy) Belief Propagation\\
EP	 	& Expectation Propagation\\
EPBP 	& Expectation Particle Belief Propagation\\
ESS 	& Effective Sample Size\\
FFBS 	& Forward Filtering Backward Smoothing \\
IPP 	& Inhomogeneous Poisson Process\\
HMM 	& Hidden Markov Model\\
MCMC 	& Markov Chain Monte Carlo\\
(L/G)MMC& Local/Global Moment Matching Conditions (\ref{eq:LMMC},~\ref{eq:GMMC})\\
MRF 	& Markov Random Field\\
NBP 	& Nonparametric Belief Propagation\\
PBP 	& Particle Belief Propagation\\
RMS(E) 	& Root Mean Squared (Error)\\
SMC 	& Sequential Monte Carlo\\
SMS 	& Sampling via Moment Sharing\\
TFS 	& Two Filter Smoothing\\ 

\end{tabular}
\setlength{\tabcolsep}{6pt} % default
\newpage

\subsubsection{Vectors, matrices and functions} 
For a vector $x\in\R^{n}$ we write $\norm{x}{p}$ the $p$-norm of $x$ with $\norm{x}{p}^{p}=\sum_{i=1:n}\abs{x_{i}}^{p}$ for $p\in[1,\infty)$. 
For $x,y\in\R^{n}$, we write $\scal{x,y}=\sum_{i=1:n}x_{i}y_{i}$ the inner product of $x$ and $y$. 
We denote the transpose of a matrix $A\in\R^{n\times n}$ by $A\transp$. $A$ is said to be semi positive definite if $\scal{x,Ax}\ge 0$ for all $x\in\R^{n}$ and positive definite if the inequality holds strictly for all $x\neq 0$. We denote by $\mathbb S_+^{n}\subset \mathbb R^{n\times n}$ the set of symmetric positive definite matrices and by $\mathbb S_-^{n}$ the set of matrices $B$ such that $-B\in\mathbb S_+^{n}$.
For a real-valued measurable function $f$ on a set $\mathcal X$, we write $\|f\|_{p}^{p}=\int_{\mathcal X} |f(x)|^{p}\dx$ the $p$-norm of $f$ for $p\in[1,\infty)$. 
We write $L^{1}(\mathcal X)$ the set of functions such that $\|f\|_{1}<\infty$ and $\mathcal P(\mathcal X) \subset L^{1}(\mathcal X)$ denotes the restriction to nonnegative functions integrating to one (probability density functions). 
For a real-valued, differentiable function $f$, we write $\nabla f$ its gradient and $\nabla^{2} f$ its Hessian provided $f$ is twice differentiable.\check{may10}

\subsubsection{Probabilities and expected values}
We will consider continuous sample spaces $\mathcal X\subseteq \R^{n}$ and, by default, we will consider the corresponding Borel $\sigma$-algebra. Correspondingly we will use the shorthand $(\mathcal X,\P)$ to denote a probability space with $\P$ the probability measure. We will consider probability measures which admit a probability density function $p$ with respect to a base measure $\nu$ with therefore $\P(x\in A)=\int_{A}p(x)\nudx$ for any open subset $A\subset \mathcal X$. For such a probability space and a vector-valued mapping $\varphi:\mathcal X\to\R^{d}$, we write the expectation and the covariance matrix of $\varphi$ respectively as $(\E_{p}[\varphi(X)])_{i}=\int_{\mathcal X}\varphi_{i}(x)p(x)\nudx$ and $\V_{p}[\varphi(X)]=\E_{p}[\varphi(X)\varphi(X)^{t}]-\E_{p}[\varphi(X)]\E_{p}[\varphi(X)]^{t}$. For two probability distribution functions $p$ and $q$ on $\mathcal X$, we write $\KL{p,q}=\E_{p}[\log p(X)]-\E_{p}[\log q(X))]$ the Kullback-Leibler divergence between $p$ and $q$.\check{june8,april27}
 
\subsubsection{Convex analysis}
Let $B\subset\R^{n}$ denote the Euclidean unit ball with $B=\{x\in\R^{n}\mid \|x\|_{2}\le 1\}$. For any subset $C\subseteq\R^{n}$, we write $\mathrm{int}\,(C)$ the interior of $C$ with $C^{\circ}=\{x\in C \mid \exists \epsilon>0, x+\epsilon B \subset C\}$. Correspondingly we write $\mathrm{cl}(C)$ the closure of $C$ with $\mathrm{cl}(C) = \cap\{C+\epsilon B\st \epsilon>0\}$. A set $C$ is convex if for any $x,y\in C$, $(1-\lambda)x+\lambda y\in C$ for all $\lambda\in(0,1)$. Let $\Omega$ be an arbitrary connected and nonempty subset of $\R^{n}$; a function $f:\Omega\to\R$ is said to be convex if $f((1-\lambda)x+\lambda y)\le (1-\lambda)f(x)+\lambda f(y)$. For an arbitrary function $f:\Omega\to\R$, we denote by $f^{\star}:\R\to \R\cup\{+\infty\}$ the convex-conjugate of $f$ with $f^{\star}(y)=\sup_{y\in \Omega}[\scal{x,y}-f(x)]$. The set of minimisers of a function $f$ on a open, nonempty set $C$ will be denoted by $\arg\min_{x\in C}\,f(x)$ i.e. the set $\pab{x\in C\mid f(y)\ge f(x), \,\forall y\in C}$. \check{may13}
%If the function $f$ is convex and differentiable on $C$ we call \emph{first-order condition} the fact that this set is equivalent to the set of points $x\in C$ such that $\nabla f(x)=0$.
